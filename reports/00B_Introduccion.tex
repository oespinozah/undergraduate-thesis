\chapter*{\centering \large INTRODUCCIÓN} % si no queremos que añada la palabra "Capitulo"
\addcontentsline{toc}{section}{INTRODUCCIÓN} % si queremos que aparezca en el índice
\markboth{INTRODUCCIÓN}{INTRODUCCIÓN} % encabezado

El manejo de la información es de carácter fundamental en la toma de decisiones. Por lo que es de vital importancia que dicha información sea confiable. Pues no basta con solo poder obtener el valor de un indicador, sino que este mismo permita reflejar el valor real en la población en la que fue medido. Más aún si el objetivo de estos indicadores es el de determinar el impacto generado por una enfermedad en una población; ya que este no se limita solo al dinero utilizado en su tratamiento, sino también engloba otro tipo de gastos, como la perdida en la esperanza y calidad de vida que conlleva padecerla. Por lo tanto, es necesario estimar indicadores como la prevalencia de forma insesgada. Por lo que de haber un sesgo de selección, es necesario establecer los pasos para corregirlo.\\
Partiendo de lo mencionado con anterioridad, la presente investigación se centró en el estudio de un método para la corrección de sesgo espacial que pueda ser aplicado en un caso real y que sirva como punto de partida en su implementación. Para esto, se empezó planteando la problemática de la situación, así como los objetivos de la investigación. Luego, se detalló en el Marco Teórico los elementos teóricos-conceptuales empleados en el desarrollo de la investigación. Después, se presentó el Marco Metodológico con el que se buscó responder a los objetivos de la investigación. Finalmente, se realizó el análisis correspondiente para las 2 poblaciones especificadas y se obtuvieron los resultados necesarios para la formulación de conclusiones acorde a las hipótesis presentadas.