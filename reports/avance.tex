\documentclass[12pt,a4paper,oneside]{report}

\usepackage[T1]{fontenc}
\usepackage{times}
\usepackage[utf8]{inputenc}
\usepackage{amsmath}
\usepackage{graphicx}
\usepackage[colorlinks=false]{hyperref}
\usepackage{multicol}
\usepackage{longtable}
\usepackage[refpages]{gloss}
\usepackage{float}
\usepackage{anysize}
\usepackage{appendix}
\usepackage{lscape}
\usepackage{pdflscape}
\usepackage{multirow}
\usepackage{listings}
\usepackage{color}
\usepackage{setspace}
\usepackage{enumerate} 

\usepackage[spanish]{babel}
\usepackage{apacite} 

\usepackage{graphicx}
\usepackage[svgnames]{xcolor}
\usepackage{booktabs}
\usepackage{lscape}
\usepackage{multirow}
\usepackage{afterpage}
\pagestyle{plain}
\usepackage{anysize}
\usepackage{here}

\usepackage{url}
\usepackage{amsmath}
\usepackage{mathtools}


\usepackage{enumitem}

\usepackage{bigints}
\newcommand\dummy{\frac{a}{c}\,\mathrm{d}P}


    
\usepackage{subcaption}

\begin{document}


Sea un  proceso de Poisson Marcado con las variables aleatorias  $X_i$ y $Y_i$ asociadas a cada una de las $N$ realizaciones de $Z_i$ (donde $Z_i=\{z_{i,1};z_{i,2}\}$) en el espacio $A$ definido en el plano $R^2$, cada uno con su respectiva función de intesidad $\lambda_\theta$ y $\lambda_\pi$. Las variables aleatorias  $X_i$ y $Y_i$ se definen como:
\begin{itemize}
    \item[] $Y_i$: Resultado de la prueba en el individuo $i$\\
    $$Y_i \sim Bernoulli(\theta_i) \; , \,Y_i = \left\{ 1: \mathrm{Positivo}\atop 0: \mathrm{Negativo} \right. $$
    $$f(Y_i=y_i|\;\theta_i) = \theta_i^{y_i} (1-\theta_i)^{1-y_i}$$
    \item[] $X_i$: Condición del individuo $i$ en la muestra\\
    $$X_i \sim Bernoulli(\pi_i) \; , \, X_i = \left\{ 1: \mathrm{Seleccionado} \atop 0: \mathrm{No \; seleccionado} \right.$$
    $$f(X_i=x_i|\;\pi_i) = \pi^{x_i} (1-\pi_i)^{1-x_i}$$
\end{itemize}
Y función de probabilidad conjunta
\begin{equation*}
    f(Y_i=y_i,X_i=x_i|\;\theta_i,\pi_i) = \theta_i^{y_i} \pi_i^{x_i} (1-\theta_i)^{1-y_i} (1-\pi_i)^{1-x_i}; \; (x,y)=\{(0,0),(1,0),(0,1),(1,1)\}
\end{equation*}
Sea
\begin{itemize}
    \item[] $\textbf{Y} = Y_1,Y_2,\dots,Y_n$
    \item[] $\textbf{X} = X_1,X_2,\dots,X_n$
\end{itemize}
Entonces las funciones de Verolimilitud (L) serían:
\begin{align*}
L(\theta;\textbf{Y}) &=  \prod_{i=1}^N \theta_i^{y_i} (1-\theta_i)^{1-y_i}\\
&= \prod_{i=1}^{N^+} \theta_i \prod_{i=N^++1}^N(1-\theta_i)\\
l(\theta;\textbf{Y}) &= \sum_{i=1}^{N^+} log (\theta_i) + \sum_{i=N^++1}^N log(1-\theta_i)
\end{align*}
\begin{align*}
L(\pi;\textbf{X}) &=  \prod_{i=1}^N \pi_i^{x_i} (1-\pi_i)^{1-x_i}\\
&= \prod_{i=1}^{n} \pi_i \prod_{i=n+1}^N(1-\pi_i)\\
l(\pi;\textbf{X}) &= \sum_{i=1}^{n} log (\pi_i) + \sum_{i=n+1}^N log(1-\pi_i)
\end{align*}
\begin{align*}
L(\theta,\pi;\textbf{Y,X}) &=  \prod_{i=1}^N \theta_i^{y_i} (1-\theta_i)^{1-y_i} \pi_i^{x_i} (1-\pi_i)^{1-x_i}\\ 
&=  \prod_{i=1}^n \theta_i^{y_i} (1-\theta_i)^{1-y_i} \pi_i \prod_{i=n+1}^N \theta_i^{y_i} (1-\theta_i)^{1-y_i}(1-\pi_i)\\
&= \prod_{i=1}^{n^+} \theta_i \pi_i \prod_{i=n^++1}^{n} (1-\theta_i) \pi_i \prod_{i=n+1}^{n+N^+-n^+} \theta_i (1-\pi_i) \prod_{i=n+N^+-n^++1}^N (1-\theta_i)  (1-\pi_i)\\
l(\theta,\pi;\textbf{Y,X}) &= \sum_{i=1}^{n^+} log( \theta_i \pi_i) +  \sum_{i=n^++1}^{n} log(1-\theta_i)\pi_i + \sum_{i=n+1}^{n+N^+-n^+} log \theta_i (1-\pi_i) \\
&+ \sum_{i=n+N^+-n^++1}^N log(1-\theta_i)  (1-\pi_i)
\end{align*}

\newpage

Si $\lambda_\theta(x) =  \lambda(x) \; \theta $
\begin{align*}
\mathrm{L}(\theta; \textbf{X}) &= \lambda_{\theta}(x_1) \lambda_{\theta}(x_2) \dots \lambda_{\theta}(x_n) \; exp\left( \int_W (1-\lambda_{\theta}(u)) \mathrm{d} u \right)\\
&= \lambda(x_1) \; \theta \; \lambda(x_2) \; \theta\dots \lambda(x_n) \; \theta \; exp\left( \int_W (1-\lambda(u) \; \theta) \mathrm{d} u \right)
\end{align*}
\begin{equation*}
   \hat{ \theta} = \cfrac{n}{\int_W \lambda(u) \mathrm{d}u }
\end{equation*}



Si $\lambda_{\hat{\theta}}(x) =  \lambda(x) \; f_\pi(x) \; \theta  $
\begin{align*}
\mathrm{L}(\hat{\theta}; \textbf{X}) &= \lambda_{\hat{\theta}}(x_1) \lambda_{\hat{\theta}}(x_2) \dots \lambda_{\hat{\theta}}(x_n) \; exp\left( \int_W (1-\lambda_{\hat{\theta}}(u)) \mathrm{d} u \right)\\
&= \lambda(x_1) \; f_\pi(x_1) \; \theta \; \lambda(x_2) \; f_\pi(x_2) \; \theta \dots \lambda(x_n) \; f_\pi(x_n)  \; \theta \\ & \; exp\left( \int_W (1-\lambda(u) \; f_\pi(u) \; \theta) \mathrm{d} u \right)
\end{align*}
\begin{equation*}
    \hat{\theta} = \cfrac{n}{\int_W \lambda(u) \; f_\pi(u) \; \mathrm{d}u }
\end{equation*}

\newpage

Sea $Y_i \sim Bernoulli(\theta)$
\begin{align*}
p(Y_i=y_i|\theta_i) &= \theta^{y_i} (1-\theta_i)^{n-y_i} \\
L(\theta|\textbf{Y}) &= \prod_{i_1}^n \theta^{y_i} (1-\theta)^{n-y_i} \\
L(\theta|\textbf{Y}) &= \prod_{i_1}^n \left(\cfrac{\theta}{1-\theta}\right) ^{y_i} (1-\theta)
\end{align*}
Si
\begin{align*}
\cfrac{\theta}{1-\theta} &= exp\left[ \; \sum_{j=1}^k \; \beta_j \;  f_j( x_{i} ) \;\right] \\
1-\theta &= \cfrac{1}{1+exp\left[ \; \sum_{j=1}^k \; \beta_j \;  f_j( x_{i} ) \;\right]}
\end{align*}
Entonces
\begin{align*}
L(\beta_1, \beta_2, \dots, \beta_k|\textbf{Y},\textbf{X}) &= \prod_{i=1}^n exp\left[ \;  \sum_{j=1}^k \; \beta_j \; f_j( x_{i} ) \;\right] ^ {y_i} \frac{1}{1+exp\left[ \; \sum_{j=1}^k \; \beta_j \; f_j( x_{i} ) \;\right]} \\
L(\mathrm{B}|\textbf{Y},\textbf{X}) &= \prod_{i=1}^n exp\left[ \;  \mathrm{B}^t \; \textbf{f}( x_{i} ) \;\right] ^ {y_i} \frac{1}{1+exp\left[ \;\mathrm{B}^t \; \textbf{f}( x_{i} ) \;\right]} \\
l(\mathrm{B}|\textbf{Y},\textbf{X}) &= \sum_{i=1}^n {y_i} \; \mathrm{B}^t \; \textbf{f}( x_{i} ) \; - \sum_{i=1}^n log \left( 1+exp\left[ \; \mathrm{B}^t \; \textbf{f}( x_{i} ) \;\right] \right)
\end{align*}
Donde $\mathrm{B}^t \; \textbf{f}( x_{i} ) =\sum_{j=1}^k \; \beta_j \; f_j( x_{i} )$

%Si
%\begin{align*}
%\cfrac{\theta}{1-\theta} &= exp\left(X_i \; \beta \right) \\
%1-\theta &= \frac{1}{1+exp\left(X_i \; \beta \right)}
%\end{align*}
%Entonces
%\begin{align*}
%L(\beta|\textbf{Y,X}) &= \prod_{i_1}^n exp\left(X_i \; \beta \right) ^ {y_i} \frac{1}{1+exp\left(X_i \; \beta \right)} \\
%l(\theta|\textbf{Y}) &= \sum_{i_1}^n {y_i} exp\left(X_i \; \beta \right) - \sum_{i_1}^n log \left( 1+exp\left(X_i \; \beta \right) \right)
%\end{align*}

\newpage

Sea un modelo de incidencia elevada $\lambda_\theta(y) = \lambda(y) \; \theta = \lambda(y) p(y=1)$. En epidemiología espacial $\lambda(y)$ es la variación espacial de la intensidad poblacional y $p(y=1)$ es el riesgo a la enfermedad por persona.
\begin{align*}
L(\theta; \textbf{Y}) &\propto \lambda_\theta(y_1) \lambda_\theta(y_2) \dots \lambda_\theta(y_n) \; exp\left( - \int_W \lambda_\theta(u) \; \mathrm{d} u \right) \\
L(\theta; \textbf{Y}) &\propto \prod_{i=1}^n \lambda_\theta(y_i)  \; exp\left( - \int_W \lambda_\theta(u) \; \mathrm{d} u \right) \\
L(\theta; \textbf{Y}) &\propto \prod_{i=1}^n \lambda(y_i) p(y_i=1)  \; exp\left( - \int_W \lambda(u) p(u=1) \; \mathrm{d} u \right)
\end{align*}
Si
\begin{equation*}
    p(y_i=1) = \cfrac{1}{1+exp\left[\; -\mathrm{B}^t \textbf{f} (x_i) \; \right]}
\end{equation*}
Entonces
\begin{align*}
L(\mathrm{B}; \textbf{Y},\textbf{X}) &\propto \prod_{i=1}^n \cfrac{\lambda(y_i)} {1+exp\left[\; -\mathrm{B}^t \textbf{f} (x_i) \; \right]} \; exp\left( - \int_W \cfrac{\lambda(u)}{1+exp\left[\; -\mathrm{B}^t \textbf{f} (u) \; \right]} \; \mathrm{d} u \right) \\
l(\mathrm{B}; \textbf{Y},\textbf{X}) &\propto \sum_{i=1}^n \lambda(y_i) - \sum_{i=1}^n \left( 1+exp\left[\; -\mathrm{B}^t \textbf{f} (x_i) \; \right] \right) - \left( \int_W \cfrac{\lambda(u)}{1+exp\left[\; -\mathrm{B}^t \textbf{f} (u) \; \right]} \; \mathrm{d} u \right)
\end{align*}


%$\cfrac{1}{1+exp\left[\; \mathrm{B}^t \textbf{f} (y) \; \right]}$

\end{document}
