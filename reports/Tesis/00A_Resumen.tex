%----------------------------------------------------------------------------------------
%	Resumen
%----------------------------------------------------------------------------------------

\chapter*{\centering \large RESUMEN} % si no queremos que añada la palabra "Capitulo"
%\addcontentsline{toc}{section}{RESUMEN} % si queremos que aparezca en el índice
\markboth{RESUMEN}{RESUMEN} % encabezado
\doublespacing

%Dada la importancia de estimar correctamente el valor para la prevalencia de una enfermedad, se propuso una metodología que permita su estimación al corregir el sesgo de selección espacial existente cuando la información se recolecta por medio de campañas en un centro de salud. 
%Se partió del supuesto de que el riesgo a ser seleccionado en la muestra no era aleatorio a nivel espacial. Con esto, se obtuv

%Pese a que los resultados en las estimación fueron no concluyentes, se analizó sobre la data real; ajustando los modelos no solo con la data esacial, sino integrando covariables epidemiológicas como factores de riesgo. Con ello se estimó sobre la población no muestreada para así buscar corregir el sesgo.
%En el caso de Corpacancha se encontró que la prevalencia se encontraba sobreestimada; mientras que en Canchayllo, esta se encontraba subestimada.
%Si bien los resultados provenientes de la simulación fueron no concluyentes, estos necesitan ser explorados en siguientes estudios integrando nuevas covariables

%----------------------------------------------------------------------------------------
%	Abstract
%----------------------------------------------------------------------------------------

\chapter*{\centering \large ABSTRACT}
%\addcontentsline{toc}{section}{ABSTRACT}
\markboth{ABSTRACT}{ABSTRACT}


%----------------------------------------------------------------------------------------
%	Prólogo
%----------------------------------------------------------------------------------------

\chapter*{\centering \large PRÓLOGO}
%\addcontentsline{toc}{section}{PRÓLOGO}
\markboth{PRÓLOGO}{PRÓLOGO}