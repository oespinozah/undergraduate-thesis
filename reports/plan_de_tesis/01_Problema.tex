\section{Planteamiento del problema}
\subsection{Descripción de la situación problemática}
El impacto generado por una zoonosis (enfermedad transmitida de animales a humanos) en la economía de una sociedad consiste no solo en el dinero invertido para su tratamiento y prevención, o de las pérdidas que esta causa en las actividades ganaderas y agrícolas. El efecto que tiene sobre las personas también incluye la discapacidad y el cambio de estilo de vida que esta y su tratamiento conlleva~\cite{shaw2017dalys}. Un caso de estas enfermerdades es la hidatidosis; la cual se da como consecuencia de una infección parasitaria que desemboca en la formación de quiste focalizados en determinados órganos como el hígado, pulmón, riñón, entre otros. En el plano internacional, se han encontrado niveles de prevalencia (porcentaje de individuos positivos a una enfermedad) del 8.1\% y 13\% de hidatidosis en hígado y pulmón, respectivamente, en zonas ganaderas del norte de Etiopía~\cite{kebede2009echinococcosis}. De manera análoga en la región de Sudamérica, durante los ultimos 40 años se han encontrado niveles que varían entre 1.6\% y 14.2\% por región~\cite{larrieu2004echinococcosis}. Estudios realizados con anterioridad en centros poblados ubicadas en áreas rurales de Junín han determinado niveles de prevalencia entre 5\% y 18\%~\cite{santivanez2010factores}, convirtiendo a este departamento en una zona altamente endémica para esta zoonosis. Pese a esto, el efecto causado por esta enfermerdad se encuentra infravalorado por la misma población~\cite{moro2011economic}, lo cual complica al desarrollo de las investigaciones relacionadas a esta durante la recolección datos.\\

Usualmente, la información relacionada a la hidatidosis es obtenida mediante un muestreo por conveniencia en el que los datos son recolectados por medio de una campaña de despitaje gratuita en el centro de salud del centro poblado que se está analizando. Este muestreo por conveniencia implica un sesgo de selección debido a que la información recopilada suele encontrarse acumulada en quienes viven en las zonas aledañas a los centros de salud, permitiendo así la presencia de un potencial error en la estimación de la prevalencia de hidatidosis que podría estar subestimando su valor real.\\

La presente investigación se centrará en plantear una metología que considere la distribución espacial de los individuos en el centro poblado respecto al centro de salud (o lugar en donde se haya realizado la campaña) en la estimación de la prevalencia buscando así corregir el problema del sesgo. Por ello, se partirá del supuesto de que la probabilidad de que un individuo de la población pertenezca a la muestra (esto es, haya sido enrolado en el estudio) será heterogénea y dependerá de la ubicación de su vivienda.


\newpage
\subsection{Formulación del Problema}
Al estar planteando una metodología, surge la necesidad de saber si esta es eficaz en comparación a la tradicional. Por esto, es necesario que la investigación pueda solucionar los siguientes problemas.
\subsubsection{Problema General}
¿Se puede emplear un método de estimación de prevalencia considerando la distribución espacial para reducir el sesgo del muestreo?
\subsubsection{Problemas Específicos}
\begin{itemize}
    \item ¿Cómo se diferencian los resultados obtenidos de las estimaciones de la prevalencia mediante el método clásico y el error de estimación del método propuesto?
    \item ¿Cuál es el método adecuado de acuerdo a las condiciones de la investigación?
    \item ¿De cuánta es la prevalencia de la hidatidosis humana en el centro poblado de Corpacancha mediante el método propuesto?
\end{itemize}
\subsection{Objetivos de la Investigación}
\subsubsection{Objetivo General}
Estudiar un método de estimación de prevalencia considerando la distribución espacial para reducir el sesgo del muestreo
\subsubsection{Objetivos Específicos}
\begin{itemize}
    \item Comparar los resultados obtenidos de las estimaciones de la prevalencia por el método tradicional y el método propuesto
    \item Determinar el método adecuado de acuerdo a las condiciones de la investigación
    \item Estimar la prevalencia de la hidatidosis humana en el centro poblado de Corpacancha mediante el método propuesto.
\end{itemize}
\newpage
\subsection{Justificación, alcances y limitaciones de la investigación}
La justificación de la presente investigación radica en poder estimar correctamente desde un marco estadístico la prevalencia de la hidatidosis con el fin de poder contar con un indicador más preciso y que puede ser empleado en la toma de decisiones. De igual forma, el alcance se encuentra en presentar una metodología que nos permita estimar eficientemente la prevalencia de hidatidosis y así contribuir en el trabajo que tiene la Organización Mundial de la Salud (OMS) respecto a validar estrategias eficaces frente al control de la hidatidosis~\cite{who2020}. Por otro lado, es necesario hacer mención de que la presente investigación se centra netamente en estimar la prevalencia, mas no en los factores asociados a esta. Además, la metodología propuesta está limitada de acuerdo a los supuestos que esta plantea conforme a las condiciones en las que la información fue recolectada. Finalmente, al contar con datos reales, la investigación se encuentran sujeta a los principios éticos para estudios médicos que incluyen sujetos humanos~\cite{general2014world}.
