\section{Metodología}
\subsection{Tipo, nivel y diseño de la investigación}
\subsubsection*{Estudio madre}
El estudio madre de donde provienen los datos es de carácter observacional y consiste en una cohorte prospectiva iniciada en el 2017 en comunidades ganaderas del departamento de Junin bajo el nombre de VIRSEL. Posterior a ello, bajo el nombre de HYCOM se hizo una segunda recolección de datos en el 2018. Finalmente, este estudio fue modificado en el 2019 ampliando su alcance al eliminar la limitante del departamento; lo cual permitió la recolección de información en la región de Huancavelica en el 2020.
\subsubsection*{Estudio de tesis}
El estudio de tesis se trata de una investigación de carácter observacional y de corte transversal con la información recolectada en el estudio madre entre los años 2017 y 2018.

\subsection{Población, muestra y tamaño de la muestra}
La población del presente estudio comprende a los habitantes del centro poblado de Corpacancha, Junín-Perú. La información recolectada (muestra) respecto a la tenencia de la enfermedad proviene de dos estudios realizados con anterioridad en la comunidad:
\begin{enumerate}
	\item VIRSEL: Estudio realizado en Octubre del 2017. La información fue recolectada por medio de una campaña de despistaje gratuita en el centro de salud del lugar
	\item HYCOM: Estudio realizado en Junio del 2018. La información fue recolectada por medio de una campaña de despistaje gratuita en la que se visitaron las casas de los habitantes
\end{enumerate}
El poder trabajar con información en relación a la tenencia de hidatidosis humana proveniente de dos estudios diferentes realizados con unos meses de diferencias es posible dado al periodo de incumbación de la enfermedad (Ver Sección \ref{hidat}). Por otro lado, en lo que respecta a la información sobre la distribución espacial de los habitantes, esta proviene de un censo realizado en la comunidad en simultaneo a los estudios VIRSEL y HYCOM. De este se tienen las coordenadas geográficas de cada casa de la comunidad.

\subsection{Técnicas de análisis e instrumentos}
Se iniciará con una limpieza y consolidación de la base de datos a fin de contar solo con la información con la que se trabajará, creando las variables necesarias para su procesamiento. Previo al procesamiento de la información real para la estimación del nivel de prevalencia de la hidatidosis humana, se realizó la simulación de múltiples escenarios tomando en consideración diferentes parámetros. Entre estos:
\begin{itemize}[noitemsep,nolistsep]
    \item Tamaño poblacional
    \item Nivel de cobertura (porcentaje de individuos de la población que participaron en la campaña de despitaje)
    \item Prevalencia de hidatidosis humana en la población
\end{itemize}

Con cada uno de los escenarios simulados, se procederá a estimar la prevalencia por medio del método tradicional y del método planteado. Cada una de estas prevalencias estimadas será comparada con el valor de la prevalencia establecido en cada simulación. Con esto se determinará bajo qué condiciones el método propuesto resulta eficiente en comparación al método tradicional tomando como criterio al Error Cuadrático Médio de las estimaciones.\\

Al pasar a la data real, la información requerida para cada método será:
\begin{enumerate}
	\item \textbf{Estimación de una proporción poblacional (método tradicional)}\\ Se empleará la información obtenida en el estudio VIRSEL por medio de la campaña de salud realizada en el Centro Poblado de Corpacancha (Junín, Perú).
	\item \textbf{Estimación de una proporción poblacional considerando la distribución espacial de los habitantes (método propuesto)}\\ Además de utilizar la información obtenida en el estudio VIRSEL por medio de la campaña de salud realizada en el Centro Poblado de Corpacancha (Junín, Perú), se usará la información proveniente del censo realizado a la comunidad para la distribución espacial de los pobladores.
\end{enumerate}

\newpage
\subsection{Cuadro de operalización de variables}

\begin{table}[htbp]
  \centering
  \caption{Cuadro de operalización de variables}
        \begin{tabular}{|p{8.785em}|p{12.5em}|p{8.57em}|p{11.215em}|}
    \toprule
    Variable de Caracterización & Definición Conceptual & Tipo de variable & Valores \\
    \midrule
    Sexo  & Sexo del individuo & Nominal & 0: Femenino\newline{}1: Másculino \\
    \midrule
    Edad  & Tiempo de vida en años del individuo & Discreta & Número entero mayor o igual a 0 \\
    \midrule
    Rango de edad & Tiempo de vida en años del individuo agrupado en rangos & Ordinal & 1: Menores de 15 años\newline{}2: De 16 a 30 años\newline{}3: De 31 a 40 años\newline{}4: De 41 a 50 años\newline{}5: De 51 a 65 años\newline{}6: De 66 años a más \\
    \midrule
    Distancia & Distancia en metros desde la casa de un individuo al centro de EsSalud & Continua & Número real mayor a 0 \\
    \midrule
    Prevalencia & Proporción de individuos que presentan la enfermedad & Continua & Número real comprendido entre 0 y 1 \\
    \bottomrule
    \end{tabular}%
  \label{tab:addlabel}%
\end{table}%


\subsection{Matriz de consistencia}
En la siguiente página encontramos la matriz de consistencia del estudio
\newpage
\begin{landscape}

\begin{table}[htbp]
  \centering
  \caption{Matriz de consistencia}
    \begin{tabular}{|p{11.07em}|p{11.07em}|r|r|p{11.07em}|}
    \toprule
    \textbf{Problema} & \textbf{Objetivos} & \multicolumn{1}{p{11.07em}|}{\textbf{Hipótesis}} & \multicolumn{1}{p{11.07em}|}{\textbf{Variables e Indicadores}} & \textbf{Metodología} \\
    \midrule
    \textbf{Problema General} & \textbf{Objetivo General} & \multicolumn{1}{p{11.07em}|}{\textbf{Hipótesis General}} & \multicolumn{1}{p{11.07em}|}{\textbf{Variable dependiente}} & \textbf{Tipo de Investigación} \\
    ¿Se puede emplear un método de estimación de prevalencia considerando la distribución espacial para reducir el sesgo del muestreo? & Estudiar un método de estimación de prevalencia considerando la distribución espacial para reducir el sesgo del muestreo & \multicolumn{1}{p{11.07em}|}{El método planteado estima la prevalencia de forma insesgada} & \multicolumn{1}{p{11.07em}|}{Prevalencia: Porcentaje de individuos que presentan la enfermedad} & Básica \\
    \textbf{Problemas Específicos} & \textbf{Objetivos Específicos} & \multicolumn{1}{p{11.07em}|}{\textbf{Hipótesis Específica}} & \multicolumn{1}{p{11.07em}|}{\textbf{Variable independiente}} & \textbf{Tipo de Investigación} \\
    ¿Cómo se diferencian los resultados obtenidos de las estimaciones de la prevalencia mediante el método clásico y el error de estimación del método propuesto? & Comparar los resultados obtenidos de las estimaciones de la prevalencia por el método tradicional y el método propuesto & \multicolumn{1}{p{11.07em}|}{El método planteado presenta un menor Error Cuadrático Medio en la estimación de la prevalencia que el método tradicional} & \multicolumn{1}{p{11.07em}|}{Ubicación de la vivienda del individuo} & Observacional \\
    ¿Cuál es el método adecuado de acuerdo a las condiciones de la investigación? & Determinar el método adecuado de acuerdo a las condiciones de la investigación & \multicolumn{1}{p{11.07em}|}{El método planteao es el adecuado para la situación que se está observando} & \multicolumn{1}{p{11.07em}|}{Presencia de la enfermedad en el individuo} & A fin de comparar los métodos antes de aplicarlos en la data real, se simularán escenario tomando en consideración diferentes condiciones \\
    ¿De cuánta es la prevalencia de la hidatidosis humana en el centro poblado de Corpacancha mediante el método propuesto? & Estimar la prevalencia de la hidatidosis humana en el centro poblado de Corpacancha mediante el método propuesto &       &       & La información del caso a tratar proviene de dos campañas médicas realizadas en la comunidad \\
    \bottomrule
    \end{tabular}%
  \label{tab:addlabel}%
  
\end{table}%

\end{landscape}
