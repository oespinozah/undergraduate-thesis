\selectlanguage{english}
\begin{poliabstract}{Abstract} 
   The impact of a disease is not limited to the expense used in the treatment or prevention of it, but also to the change in the lifestyle of infected people and the economic loss that it brings as a consequence in their locality. Therefore, it is essential to correctly determine the prevalence (percentage of inhabitants with a certain disease) of this. However, in neglected diseases such as hydatidosis, it can be found that there is a bias in its estimation; caused by conditions in how the information was collected. Under this problem, the present investigation will focus on proposing a methodology that seeks to reduce this bias by means of the spatial distribution modeled on the basis of the heterogeneous probability that each individual has of being enrolled in the study.\\
   
   \textit{\textbf{Keywords:}} Prevalence, selection bias, spatial distribution
   
\end{poliabstract}
 
\selectlanguage{spanish}
\begin{poliabstract}{Resumen}
   
   El impacto de una enfermedad no se limita al gasto que se emplea en el tratamiento o prevension de esta, sino también al cambio en el estilo de vida de las personas infectadas y a la pérdida económica que trae como consecuencia en su localidad. Por ello, es fundamental determinar correctamente la prevalencia (porcentaje de habitantes con determinada enfermedad) de esta. No obstante, en enfermedades desatendidas como la hidatidosis se puede encontrar que hay un sesgo en su estimación; causado por las condiciones en como la información fue recolectada. Bajo esta problemática, la presente investigación se centrará en plantear una metodología que busque disminuir este sesgo por medio de la distribución espacial modelada a base de la probabilidad heterogénea que cada individuo tiene de ser enrolado en el estudio.\\
   
   \textit{\textbf{Palabras clave:}} Prevalencia, sesgo de selección, distribución espacial
   
\end{poliabstract}
